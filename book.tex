\documentclass{book}
\usepackage{amsmath,amsfonts,amssymb,amsthm}
\title{A Programmer's Guide to Math}
\author{Eli Sadoff}
\date{2016}
\begin{document}
\maketitle

\chapter[0]{Introduction}
Hello, world! If you bought this book it is likely that you fall into one of two categories.
The first being that you are a programmer who wants to improve their math skills, or the other
being that you are a math person who wants to better understand how math can be applied to 
programming. If you fall into either of these categories I hope that this book works out very
well for you because the book is mostly made for you guys. This book requires some understanding 
of programming concepts. The book itself will show code examples in {\tt Ruby}; however, there 
will be an appendix with each piece of code also translated into: {\tt C}, {\tt Java}, 
{\tt JavaScript}, {\tt Python}, and {\tt MatLab}.

{\bf A Note on Programming Languages}: Optimally, I would prefer this book to be language
agnostic; however, there are two issues with this. The first being that I want to show specific
implementations that can be run on the computer. While I could theoretically provide Pseudo-code 
for these implementations and allow the reader to interpret the pseudo-code into the language 
of choice for the reader, I fear that some people might not be able to translate the pseudo-code
into their language, or at least not do so properly. In addition, this book will allow the reader
to see specific advantages and disadvantages with implementation of certain algorithms in certain
languages. 

Without further ado, let's begin.

\end{document}